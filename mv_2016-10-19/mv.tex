\documentclass[a4paper,12pt,titlepage]{scrartcl}

%Pakete
%\usepackage[left=3cm,right=2cm,top=2cm,bottom=3cm]{geometry} %Seitenränder
\usepackage[utf8]{inputenc} % ermöglicht die direkte Eingabe der Umlaute
\usepackage [german]{babel} %Spracheinstellungen

\usepackage {csquotes} % Anführungszeichen nach dem Stil \enquote{ich bin zitiert.}
\usepackage{subscript} % erlaubt Tiefstellen von Zahlen und Text
\usepackage{tabularx} %krassere tabellen
\usepackage{datetime}
\usepackage{needspace}
\usepackage{tikz}
\usepackage{eurosym}
\sloppy %macht ungefähren Blocksatz, wenn nichts anderes an Trennhilfen was nützt

% Hurenkinder und Schusterjungen verhindern
\clubpenalty10000
\widowpenalty10000
\displaywidowpenalty=10000

\def \logo {
	\begin{picture}(0,0)
		\put(-9, 2){\parbox{180mm}{
				%% Inline Logo
				\begin{tikzpicture}[
						y=0.8pt,
						x=0.8pt,
						yscale=-1,
						inner sep=0pt,
						outer sep=0pt,
						scale=0.2
					]
					\begin{scope}[shift={(-448.05262,-67.70133)}]
						\path[color=black,fill=black,even odd rule,line width=16.000pt]
						(534.1205,161.8816) -- (534.1205,172.2877) -- (581.3732,172.2877) .. controls
						(571.8452,181.8308) and (562.3036,191.3603) .. (552.7463,200.8741) --
						(565.3389,200.8741) .. controls (585.4511,200.8741) and (601.7400,217.2036) ..
						(601.7400,237.3158) .. controls (601.7400,257.4280) and (585.4511,273.7170) ..
						(565.3389,273.7170) .. controls (545.2267,273.7170) and (528.8972,257.4280) ..
						(528.8972,237.3158) -- (518.4911,237.3158) .. controls (518.4911,263.1743) and
						(539.4804,284.1231) .. (565.3389,284.1231) .. controls (591.1974,284.1231) and
						(612.1462,263.1743) .. (612.1462,237.3158) .. controls (612.1462,215.3233) and
						(596.9939,196.8355) .. (576.5548,191.8042) -- (606.5180,161.8816) -- cycle;
						\begin{scope}[cm={{-1.0,0.0,0.0,-1.0,(1130.6373,475.18252)}}]
							\path[color=black,fill=black,even odd rule,line width=16.000pt]
							(565.3389,208.6889) .. controls (549.5365,208.6889) and (536.7120,221.5134) ..
							(536.7120,237.3158) .. controls (536.7120,249.4467) and (544.2722,259.8031) ..
							(554.9328,263.9587) -- (554.9328,252.2164) .. controls (550.2090,248.9346) and
							(547.1181,243.5056) .. (547.1181,237.3158) .. controls (547.1181,227.2597) and
							(555.2828,219.0950) .. (565.3389,219.0950) .. controls (575.3950,219.0950) and
							(583.5192,227.2597) .. (583.5192,237.3158) .. controls (583.5192,243.4955) and
							(580.4441,248.9328) .. (575.7450,252.2164) -- (575.7450,263.9587) .. controls
							(586.3945,259.8031) and (593.9253,249.4467) .. (593.9253,237.3158) .. controls
							(593.9253,221.5134) and (581.1413,208.6889) .. (565.3389,208.6889) -- cycle;
							\path[color=black,fill=black,even odd rule,line width=16.000pt]
							(565.3389,232.0925) .. controls (562.4657,232.0925) and (560.1156,234.4426) ..
							(560.1156,237.3158) -- (560.1156,268.4937) -- (570.5217,268.4937) --
							(570.5217,237.3158) .. controls (570.5217,234.4426) and (568.2121,232.0925) ..
							(565.3389,232.0925) -- cycle;
						\end{scope}
						\path[fill=black] (630.8529,149.4914) .. controls (625.1306,149.5458) and
						(620.4890,150.3447) .. (619.3535,151.0301) .. controls (617.0825,152.4009) and
						(616.5461,155.6877) .. (617.2480,156.4558) .. controls (617.9500,157.2237) and
						(628.6785,164.1143) .. (633.3633,166.3761) .. controls (645.9778,172.4665) and
						(632.0756,195.7720) .. (621.2161,191.6018) .. controls (613.5199,188.6464) and
						(604.7892,179.8446) .. (603.3192,179.7785) .. controls (601.8492,179.7137) and
						(599.9557,180.7048) .. (599.4726,183.2607) .. controls (598.9895,185.8166) and
						(602.4810,198.1576) .. (611.2554,207.7576) .. controls (621.7846,219.2778) and
						(638.1817,206.5491) .. (638.1817,217.0704) --
						(638.1817,265.9427)arc(179.961:0.039:13.018) -- (664.2172,203.5060) --
						(664.2172,188.9294) .. controls (664.2172,183.4623) and (670.9201,161.5406) ..
						(648.7498,152.3663) .. controls (643.3754,150.1424) and (636.5752,149.4370) ..
						(630.8529,149.4914) -- cycle;
					\end{scope}
					\path[color=black,fill=black,even odd rule,line width=8.333pt]
					(150.0000,23.1213) .. controls (79.9291,23.1213) and (23.1213,79.9291) ..
					(23.1213,150.0000) .. controls (23.1213,220.0708) and (79.9291,276.8787) ..
					(150.0000,276.8787) .. controls (220.0708,276.8787) and (276.8787,220.0708) ..
					(276.8787,150.0000) .. controls (276.8787,79.9291) and (220.0708,23.1213) ..
					(150.0000,23.1213) -- cycle(150.0000,40.5068) .. controls (210.4603,40.5068)
					and (259.4540,89.5397) .. (259.4540,150.0000) .. controls (259.4540,210.4603)
					and (210.4603,259.4540) .. (150.0000,259.4540) .. controls (89.5278,259.8428)
					and (41.3547,203.5210) .. (40.5068,150.0000) .. controls (40.5068,89.5397) and
					(89.5397,40.5068) .. (150.0001,40.5068) -- cycle;
				\end{tikzpicture}
				%% Logo End
		}}
	\end{picture}
}

%\usepackage[final]{graphicx}
\usepackage{floatflt}

\addto\captionsgerman{%
  \renewcommand{\contentsname}%
    {Tagesordnung}%
}

\def \thetitle {Mitgliederversammlung des Netz39 \nolinebreak e.\,V.}

\title{ \logo \\ \vspace{0.2\baselineskip} \thetitle}
\author{
Leitung: Sebastian Mai\\
Protokollant/Schriftführer: Robert Bergner \\
Ort der Versammlung:\\ Netz39 e.V., Leibnizstraße 32, 39104 Magdeburg \\
}
\date{\displaydate{date}} % sollte man u.U. anpassen ;)

\usepackage{fancyhdr}
\setlength{\headheight}{3\baselineskip}
\pagestyle{fancy}
\renewcommand{\headrulewidth}{0pt}

\fancyhead[EL,OL]{
	\logo
}
\fancyhead[EC,OC] {
	\thetitle}
\fancyhead[ER,OR] {
	\displaydate{date}
}
\newdate{date}{19}{10}{2016}

\usepackage[colorlinks=true,linkcolor=black,citecolor=black,urlcolor=black,breaklinks=true]{hyperref}
\usepackage[colorinlistoftodos]{todonotes}

\begin{document}
\maketitle
\tableofcontents

\clearpage

\section{Begrüßung und Feststellung der Beschlussfähigkeit}
Der stellvertrende Vorsitzende des Vereins (Sebastian Mai) begrüßt die zur Versammlung anwesenden Mitglieder in den Räumen des Netz39 e.\,V.

\subsection{Feststellung der Beschlussfähigkeit}
Der Verein hat zum Zeitpunkt der Mitgliederversammlung 51 Mitglieder. Es sind 19 von 51 stimmberechtigten Mitgliedern des Vereins. Die Tagesordnung (siehe Anhang) wurde mit der Einladung versandt. Es wurde satzungsgemäß eingeladen. Die Beschlussfähigkeit wird festgestellt.

\subsection{Beschluss über Änderungsanträge zur Tagesordnung}
Update zum Änderungsantrag in TOP 5.

\section{Bericht über die Tätigkeit des Vorstands}
Der Vorstand berichtet über Formalien und Tagesgeschäft der letzten Amtszeit von Oktober 2015 bis zum Zeitpunkt der Mitgliederversammlung.

Während der Amtszeit des letzten Vorstands kam es am 25.08.2015 zu einer Bestellung bei der Firma Elgros GmbH eines Beamers zur Nutzung in den Vereinsräumen. Dabei wurde ein Beamer bestellt, aber nicht geliefert. Ein Rücktritt vom Kaufvertrag wurde mithilfe eines Widerufs bestätigt. Allerdings wurde das Geld nicht zurücküberwiesen. Es entstand eine Schadenssumme von \EUR{650}. 
Am Anfang der Vorstandsarbeit stand der Rechtsstreit um diese Schadenssumme mit der Verkäufer. Dabei wurde am 01.12.2015 ein Mahnbescheid abgeschickt, der \EUR{32} weitere Kosten verursachte. Als keine Antwort kam, wurde am 12.02.2016 ein Vollstrackungsbescheid in Höhe von \EUR{711,32} an die Elgros GmbH geschickt. Die Antwort vim Obergerichtsvollzieher war: ''Den gestellten Pfandungsauftrag habe ich gem. §532 GVGA eingestellt, da eine Vollstreckung gegen die Schuldnerin aussichtslos erscheint. Es handelt sich hier um eine reine Briefkastenadresse. D. Schu. hat hier nur ein virtuelles Office-Buro wo er nur Telefonate und Briefe entgegennehmen lasst.''
Die Kosten in Höhe von \EUR{61,05} des Gerichtsvollziehers wurden dem Verein in Rechung gestellt. Am 30.06.2016 wurde durch den Vorstand eine Strafanzeige gegenber der Elgros GmbH gestellt. Alle nötigen Unterlagen wurden der Polizei übermittelt. Die telefonische Nachfrage bei der Polizei war bisher nicht erfolgreich, da sich die Polizei im Umzug befindet.

Neben dem Rechtsstreit kümmerte sich der Vorstand um die Sortierung und Aufbereitung der Mitgliederdaten, Rechnung und des Briefverkehrs. Dabei empfiehlt der Vorstand dem neuen Vorstand den neu ankommenden Schriftverkehr auszudrucken und abzuheften um aufkommende Rechtsstreite besser führen zu können.

Zudem bereitete der Vorstand die Dokumente für die Bestätigung und Verlängerung der Allgemeinnützigkeit des Netz39. Um dies fertigzustellen muss noch die Einnahmenüberschussrechnung (EÜR) erstellt werden. Diese kann aber laut Carina Lederich aus der Buchhaltungssoftware exportiert werden.

Außerdem hat ein Mitglied seit Anfang des Jahres keine Mitgliedsbeitrge mehr bezahlt. Auf den bekannten Kommunikationskanälen konnte diese Person jedoch nicht erreicht werden. Der Ausschlussantrag fehlt jedoch auf der Tagesliste, weswegen in dieser Mitgliederversammlung, dieses Mitglied im Verein verbleibt.

\section{Entlastung des Vorstands}

\subsection{Bericht der Kassenprüfer}
Es wurde eine Kassenprüfung am 16.09.2016 durchgeführt.
Ergebnis der Kassenprüfung: Der aktuelle Kontostand ist \EUR{4451,32}. Die Gesamtspendenhöhe ist ... . Bei der Kassenprüfung wurde jedoch bemerkt, dass 1 Bubblecom-Posten von \EUR{19,95} nicht zurückgebucht werden konnte und dieser Betrag als Verlust beim Netz32 e.V. verbleibt. Alle weiteren Abbuchungen wurden korrekt abgewickelt. 
Die Kassenprüfer Uwe Hermsdorf und Katharina Lehmann empfehlen eine Entlastung der Schatzmeisterin Carina Lederich.

\subsection{Entlastung des Vorstandes}
\begin{tabularx}{\textwidth}[b]{l | l | X | X | X}
	Vorstandsmitglied & Posten & 	Dafür & Dagegen & Enthaltungen \\
	\hline
	Michel Vorsprach & Vorstandsvorsitzender & 17 & 0 & 1 \\
	Sebastian Mai & Stellvertretender Vorsitzender & 16 & 0 & 2 \\
	Carina Lederich & Schatzmeister & 16 & 0 & 2 \\
\end{tabularx} \\ \\
Die Mitgliederversammlung beschließt, den bisherigen Vorstand zu entlasten. Eine Kassenprüfung fand statt.

\section{Beschlussfassung über einen Änderungsantrag zur \enquote{Ermöglichung eines sofortigen Vereinsaustritts} nach https://github.com/netz39/Ordnungen/pull/1/}
\subsection{Vorstellung es Satzungänderungsantrag 1.1}

\section{Wahlen}

Zunächst wird erklärt, welche Aufgaben der neue Vorstand erfüllen soll und wie die Wahl abläuft. Wahlleiter ist Michel Vorsprach (einstimmig beschlossen).

\subsection{Wahl des Vorstandsvorsitzenden und stellvertretenden Vorsitzenden}
Der Vorstandsvorsitzende wird in offener Wahl gewählt.
Nominiert für Vorsitzenden sind Sebastian Heerwald und Tatjana Ruhl.
Die Auszählung ergab: 12 Stimmen für Sebastian Heerwald, 5 Stimmen für Tatjana Ruhl, 0 Enthaltungen. Sebastian Heerwald nimmt die Wahl zum Vorstandsvorsitzenden des Netz39 an.

Nominiert für den Posten des stellvertretenden Vorsitzenden ist Tatjana Ruhl.
Mit 16 Stimmen dafür und einer Enthaltung wird Tatjana Ruhl zum Stellvertretenden Vorsitzenden Gewählt.
Sie nimmt die Wahl an.

\subsection{Wahl des Schatzmeisters}
Nominiert für den Posten des Schatzmeisters ist Oskar Kirmis. Er wird mit 16 Stimmen dafür und einer Enthaltung gewählt.
Oskar Kirmis nimmt die Wahl an.

\subsection{Wahl der Kassenprüfer}
Nominiert sind XXX und XXX. \\
\begin{tabularx}{\textwidth}[b]{l | X | X }
	Name & 1. Wahlgang & 2. Wahlgang \\ \hline
	XXX &  &  \\
	XXX &  & \\
	Enthaltungen & 0 & 0
\end{tabularx} \\ \\

Im ersten Wahlgang wird XXX zum Kassenprüfer gewählt, im zweiten Wahlgang wird XXX gewählt.
Beide nehmen die Wahl an.

\nopagebreak
\vspace{10\baselineskip}
\begin{tabularx}{\textwidth}[b]{X X}
	\hline
	Robert Bergner & Sebastian Heerwald \\
	Protokollant & Vorsitzender
\end{tabularx}

\appendix
Anhang:

Angaben über die gewählten Vorstandsmitglieder.

\subsubsection*{Vorsitzender}
René Meye\\
Geboren am: 28.05.1988\\
Straße: Leipziger Str. 02\\
PLZ, Ort: 39112 Magdeburg\\

\subsubsection*{Stellvertretender Vorsitzender}
Sebastian Mai\\
Geboren am: 24.09.1990\\
Straße: Umfassungsstraße 42\\
PLZ, Ort: 39124 Magdeburg\\

\subsubsection*{Schatzmeister}
David Kilias\\
Geboren am: 29.11.1986\\
Straße: Pappelallee 20\\
PLZ, Ort: 39106 Magdeburg\\


\end{document}
